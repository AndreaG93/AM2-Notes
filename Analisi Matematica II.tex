\documentclass[10pt,a4paper]{article}
\usepackage[utf8]{inputenc}
\usepackage[italian]{babel}
\usepackage{amsmath}
\usepackage{amsfonts}
\usepackage{amssymb}
\author{Andrea Graziani}
\title{Formulario}


\begin{document}

\section{Integrali multipli e curvilinei}

\subsubsection{Coordinate}

\begin{description}
\item[Coordinate polari]

\[
\begin{cases}
x = x_0 + \rho \cos \theta \\
y = y_0 + \rho \sin \theta \\
\end{cases}
\]

\[ \left| \dfrac{\partial(x,y)}{\partial(u,v)} \right| = \rho \]

\item[Coordinate ellittiche]

\[
\begin{cases}
x = x_0 + \rho a\cos \theta \\
y = y_0 + \rho b\sin \theta \\
\end{cases}
\]

\[ \left| \dfrac{\partial(x,y)}{\partial(u,v)} \right| = ab\rho \]

L'equazione canonica ellisse:
\[
\dfrac{(x-x_0)^{2}}{a^2} - \dfrac{(y-y_0)^{2}}{b^2} = 1
\]

\end{description}

\section{Analisi complessa}

\begin{equation}
\oint_\gamma f(z)dz = 2\pi i \sum_{k = 1}^{n} n(\gamma, z_{k})Res(f, z_{k})
\end{equation}


\begin{equation}
Res(f, z_{0}) = \dfrac{1}{(m-1)!} \left| \dfrac{d^{m-1}}{dz^{m-1}}\left[ (z-z_{0})^{m}f(z) \right]  \right|_{z=z_{0}}
\end{equation}

Se $f(z)=\dfrac{g(z)}{h(z)}$ con $g$ e $h$ olomorfe e $z_0$ una singolarità di ordine 1 allora:

\begin{equation}
Res(f, z_{0}) = \dfrac{g(z_0)}{h'(z_0)}
\end{equation}



\subsection{Calcolo di integrali reali con i residui}

\begin{enumerate}
\item Siano $P(x)$ e $Q(x)$ due polinomi tali che $grado(Q(x)) - grado(P(x)) > 1$ con $Q(x)\neq 0$ per ogni $x \in \mathbb{R}$. Vale che:

\begin{equation}
\int_{\infty}^{\infty} \dfrac{P(x)}{Q(x)}dx = 2\pi i \sum_{k = 1}^{n} Res(f, z_{k})
\end{equation}

Dove le singolarità $z_k$ sono tali che $Im(x_k)>0$ per ogni $k = 0,1,2 \dotsc n-1$

\item Sia dato il seguente integrale:

\begin{equation}
\int_{0}^{2\pi} g(\sin \theta, \cos \theta)d\theta
\end{equation}

Sostituendo:
\[\sin \theta = \dfrac{z - \dfrac{1}{z}}{2i}\]
\[\cos \theta = \dfrac{z + \dfrac{1}{z}}{2}\]

Risulta che:

\begin{equation}
\int_{0}^{2\pi} g(\sin \theta, \cos \theta)d\theta = \int_{|z|=1} f(z) \dfrac{dz}{iz} = 2\pi i \sum_{k = 1}^{n} Res \left( \dfrac{f(z)}{iz}, z_{k} \right)
\end{equation}

Dove le singolarità $z_k$ sono tali che $|(x_k)|<1$ per ogni $k = 0,1,2 \dotsc n-1$


\item 

\begin{equation}
\int_{\infty}^{\infty} \dfrac{P(x)}{Q(x)} \cos(ax)dx = Re \left( 2\pi i \sum_{k = 1}^{n} Res \left( \dfrac{P(x)}{Q(x)}e^{aiz}, z_{k} \right) \right)
\end{equation}

\begin{equation}
\int_{\infty}^{\infty} \dfrac{P(x)}{Q(x)} \sin(ax)dx = Im \left( 2\pi i \sum_{k = 1}^{n} Res \left( \dfrac{P(x)}{Q(x)}e^{aiz}, z_{k} \right) \right)
\end{equation}

Dove le singolarità $z_k$ sono tali che $Im(x_k)>0$ per ogni $k = 0,1,2 \dotsc n-1$

\end{enumerate}






\section{Trasformata di Laplace}


\subsection{Trasformate di funzioni elementari}

\begin{equation}
\mathcal{L}[1](s) = \dfrac{1}{s}
\end{equation}

\begin{equation}
\mathcal{L}[e^{at}](s) = \dfrac{1}{s-a}
\end{equation}

\begin{equation}
\mathcal{L}[\sin (at)](s) = \dfrac{a}{s^{2}+a^{2}}
\end{equation}

\begin{equation}
\mathcal{L}[\cos (at)](s) = \dfrac{s}{s^{2}+a^{2}}
\end{equation}

\begin{equation}
\mathcal{L}[\sinh (at)](s) = \dfrac{a}{s^{2}-a^{2}}
\end{equation}

\begin{equation}
\mathcal{L}[\cosh (at)](s) = \dfrac{s}{s^{2}-a^{2}}
\end{equation}

\begin{equation}
\mathcal{L}[t^{n}](s) = \dfrac{n!}{s^{n+1}}
\end{equation}

\begin{equation}
\mathcal{L}[\delta(t-t_{0})](s) = e^{-st_{0}}
\end{equation}

\subsection{Proprietà}

\begin{equation}
\mathcal{L}[e^{at}f(t)](s) = F(s-a)
\end{equation}

\begin{equation}
\mathcal{L}[H(t-a)f(t-a)](s) = e^{-as}F(s)
\end{equation}

\begin{equation}
\mathcal{L}[f(at)](s) = \dfrac{1}{a}F\left( \dfrac{s}{a}\right)
\end{equation}

\begin{equation}
\mathcal{L}[t^{n}f(t)](s) = (-1)^{n}\dfrac{d^{n}}{ds^{n}}(F(s))
\end{equation}

\begin{equation}
\mathcal{L}[f^{(n)}(s)](s) = s^{n}F(s) - s^{n-1}f_{+}(0^{+}) - s^{n-2}f_{+}'(0) \dotsm - f^{(n-1)}(0)
\end{equation}

\begin{equation}
\mathcal{L}\left[\int_0^t f(t) dx\right] = \dfrac{F(s)}{s}
\end{equation}

\begin{equation}
f(t) = \mathcal{L}^{-1}[F(t)] = \sum_{k = 1}^{n} Res(F(s)e^{st}, s_{k})
\end{equation}

\begin{equation}
\mathcal{L}[f * g] = \mathcal{L}[f]\mathcal{L}[g]
\end{equation}

\end{document}